\documentclass[12pt]{article}

\usepackage{parskip,palatino,amsthm,amsmath,amsfonts,amssymb}
\usepackage{multicol}
\usepackage{enumerate}
\usepackage[letterpaper,margin=1in,bottom=0.7in]{geometry}

\newcommand{\dd}[2]{\frac{d #1}{d #2}}
\newcommand{\pd}[2]{\frac{\partial #1}{\partial #2}}
\newcommand{\brf}[2]{\left(\frac{#1}{#2}\right)}
                       % Bracket-frac, e.g. for (n\pi x/L) in Fourier series
\newcommand{\fsin}[1]{\sin\brf{#1 \pi x}{L}}
\newcommand{\fcos}[1]{\cos\brf{#1 \pi x}{L}}
\newcommand{\RR}{\mathbf{R}}
\newcommand{\CC}{\mathbf{C}}
\newcommand{\ZZ}{\mathbf{Z}}
\newcommand{\mk}{\mathfrak}
\newcommand{\mb}{\mathbf}
\newcommand{\OP}{\operatorname}
\newcommand{\MATR}[9]{\left(\begin{array}{cccc}#1 & #2 & \cdots & #3\\ #4 & #5 & \cdots & #6\\ \vdots & \vdots & \ddots & \vdots\\ #7 & #8 & \cdots & #9\end{array}\right)}
\newcommand{\matr}[4]{\left(\begin{array}{cc}#1 & #2\\ #3 & #4\end{array}\right)}
\newcommand{\matt}[9]{\left(\begin{array}{ccc}#1 & #2 & #3\\#4 & #5 & #6\\#7 & #8 & #9\end{array}\right)}
\newcommand{\vect}[3]{\left(\begin{array}{c}#1 \\ #2 \\ #3\end{array}\right)}
\newcommand{\ad}{\OP{ad}}

\newtheorem{thm}{Theorem}
\newtheorem{lma}[thm]{Lemma}

\theoremstyle{definition}
\newtheorem{question}{Question}
\newtheorem{answer}{Answer}

\theoremstyle{remark}
\newtheorem*{rmk}{Remark}

%%%%%%%%%%%%%%%%%% Add extra space before theorems

\begingroup 
\makeatletter 
\@for\theoremstyle:=definition,remark,plain,TheoremNum\do{% 
\expandafter\g@addto@macro\csname th@\theoremstyle\endcsname{% 
\addtolength\thm@preskip\parskip 
}% 
} 
\endgroup 

%\include{diagrams}

\title{Sheet 6: More on representations}
\author{J. Evans}
\date{}

\begin{document}
\maketitle

\begin{question}\ \\
Write out the action of $X,Y\in\mathfrak{sl}(2,\CC)$ on $\OP{Sym}^3(\CC^2)$ explicitly.
\end{question}

\iffalse
\begin{answer}
The $\OP{Sym}^3\CC^2$ action of $X$ with respect to the basis
\[e_1^{\otimes 3},\ e_1^2e_2=\frac{1}{3}(e_1\otimes e_1\otimes e_2+\mbox{cyclic permutations}),\ e_1e_2^2=\frac{1}{3}(e_1\otimes e_2\otimes e_2+\mbox{cyclic permutations}),\ e_2^3\]
sends
\begin{align*}
e_1^{\otimes 3}&\mapsto 0\\
e_1^2e_2&\mapsto e_1^{\otimes 3}\\
e_1e_2^2&\mapsto 2e_1^2e_2\\
e_2^{\otimes 3}&\mapsto 3e_1e_2^2
\end{align*}
and for $Y$ we get
\begin{align*}
e_1^{\otimes 3}&\mapsto 3e_1^2e_2\\
e_1^2e_2&\mapsto 2e_1e_2^2\\
e_1e_2^2&\mapsto e_2^{\otimes 3}\\
e_2^{\otimes 3}&\mapsto 0.
\end{align*}
As matrices with respect to this polynomial basis, we have
\[\OP{Sym}^3(X)=\left(\begin{array}{cccc}0&1&0&0\\ 0&0&2&0\\ 0&0&0&3\\ 0&0&0&0\end{array}\right)\]
and
\[\OP{Sym}^3(Y)=\left(\begin{array}{cccc}0&0&0&0\\ 3&0&0&0\\ 0&2&0&0\\ 0&0&1&0\end{array}\right)\]
\end{answer}
\newpage
\fi

\bigskip

\begin{question}\ \\
Decompose the following representations of $\mathfrak{sl}(2,\CC)$ into irreducible summands:
\begin{multicols}{2}
\begin{enumerate}[(a)]
\item $\Lambda^2\OP{Sym}^3\CC^2$,
\item $\OP{Sym}^2\OP{Sym}^2\CC^2$,
\item $\Lambda^3\OP{Sym}^4\CC^2$,
\item $\OP{Sym}^3\OP{Sym}^2\CC^2$,
\item $\OP{Sym}^2\OP{Sym}^3\CC^2$,
\item $\OP{Sym}^2\Lambda^2\OP{Sym}^3\CC^2$,
\item $\OP{Sym}^2\OP{Sym}^4\CC^2$,
\item $\OP{Sym}^3\OP{Sym}^4\CC^2$,
\end{enumerate}
\end{multicols}
From your computations in (g) and (h) deduce that there are quadratic and cubic invariants $g_2(a,b,c,d,e)$ and $g_3(a,b,c,d,e)$ for binary quartic polynomials $ax^4+bx^3y+cx^2y^2+dxy^3+ey^4$ under the action of $SL(2,\CC)$.
\end{question}

\iffalse
\begin{answer}
In all cases the method is the same: write out a basis of vectors with fixed weights and then appeal to the classification of irreducible representations. In each case, $v$ will denote a highest weight vector for $\OP{Sym}^m\CC^2$ and $e_n$ will denote $Y^nv$. Recall that $XY^nv=(m-n+1)nY^{n-1}v$ so $Xe_n=(m-n+1)ne_{n-1}$ and $Ye_n=e_{n+1}$.
\begin{enumerate}[(a)]
\item We have a basis $e_0,e_1,e_2,e_3$ for $\OP{Sym}^3\CC^2$ and a basis $e_0\wedge e_1$, $e_0\wedge e_2$, $e_0\wedge e_3$, $e_1\wedge e_2$, $e_1\wedge e_3$, $e_2\wedge e_3$ for $\Lambda^2\OP{Sym}^3\CC^2$. Since $e_n$ has weight $2n-3$ and $e_i\wedge e_j$ has weight $2i+2j-6$, these terms each transform with weight $-4,-2,0,0,2,4$. There is therefore an irreducible subrepresentation with weight $4$, isomorphic to $\OP{Sym}^4\CC^2$ by the classification of irreducible representations. This accounts for all but a one-dimensional subrepresentation which is therefore trivial, so $\Lambda^2\OP{Sym}^3\CC^2\cong\OP{Sym}^4\CC^2\oplus\CC$. Note that we can identify the trivial summand: it is spanned by a linear combination $\alpha=Ae_0\wedge e_3+Be_1\wedge e_2$ which is annihilated by $X$ (and $Y$). Since $Xe_3=3e_2$, $Xe_2=4e_1$, $Xe_1=3e_0$, $Xe_0=0$, we have
\[X\alpha=3Ae_0\wedge e_2+3Be_0\wedge e_2\]
so the one-dimensional subrepresentation is spanned by $e_0\wedge e_{3}-e_{1}\wedge e_2$.
\item Taking a basis $e_0,e_1,e_2$ of $\OP{Sym}^2\CC^2$, forming all possible homogeneous quadratic polynomials in these basis elements and grouping them according to weights gives
\[e_0^2\qquad e_0e_1,\qquad e_0e_2,\ e_1^2,\qquad e_1e_2,\ e_2^2\]
(note that we are writing $e_ie_j$ for $\frac{1}{2}(e_i\otimes e_j+e_j\otimes e_i)$) so that, by the classification of irreducible representations, the representation $\OP{Sym}^2\OP{Sym}^2\CC^2$ splits as $\OP{Sym}^4\CC^2\oplus\CC$. In this instance, the trivial subrepresentation is spanned by $e_1^2-4e_0e_2$. To see this, note that $X(e_1^2)=(Xe_1)\otimes e_1+e_1\otimes(Xe_1)=2(e_0\otimes e_1+e_1\otimes e_0)=4e_0e_1$ while $X(e_0e_2)=e_0e_1$.
\item By arguing similarly we get $\Lambda^3\OP{Sym}^4\CC^2\cong\OP{Sym}^6\CC^2\oplus\OP{Sym}^2\CC^2$.
\item By arguing similarly we get $\OP{Sym}^3\OP{Sym}^2\CC^2\cong\OP{Sym}^6\CC^2\oplus\OP{Sym}^2\CC^2$.
\item By arguing similarly we get $\OP{Sym}^2\OP{Sym}^3\CC^2\cong\OP{Sym}^6\CC^2\oplus\OP{Sym}^2\CC^2$.
\item In this example we already know from the first part that $\Lambda^2\OP{Sym}^3\CC^2=\OP{Sym}^4\CC\oplus\CC$ so we take bases $e_0,e_1,e_2,e_3,e_4$ of $\OP{Sym}^4\CC^2$ and $f$ of $\CC$, so a basis for the symmetric-square is
\[e_ie_j,\ 0\leq i\leq j\leq 4,\qquad fe_i,\ 0\leq i\leq 4,\ \qquad f^2\]
and we compute that the irreducible decomposition is $\OP{Sym}^8\CC^2\oplus 2\OP{Sym}^4\CC^2\oplus2\CC$.
\item We get $\OP{Sym}^2\OP{Sym}^4\CC^2=\OP{Sym}^8\CC^2\oplus\OP{Sym}^4\CC^2\oplus\CC$.
\item We get $\OP{Sym}^3\OP{Sym}^4\CC^2=\OP{Sym}^{12}\CC^2\oplus\OP{Sym}^8\CC^2\oplus\OP{Sym}^6\CC^2\oplus\OP{Sym}^4\CC^2\oplus\CC$.
\end{enumerate}
There is a unique quadratic invariant $g_2$ because $\OP{Sym}^2\OP{Sym}^4\CC^2$ has a unique one-dimensional trivial subrepresentation. Similarly for $g_3$.
\end{answer}
\newpage
\fi

\bigskip

\begin{question}(Clebsch-Gordan theorem)\\
Let $V$ denote the standard 2-dimensional representation of $\mathfrak{sl}(2,\CC)$. Prove that the tensor product $\OP{Sym}^m(V)\otimes\OP{Sym}^n(V)$ decomposes into irreducible representations
\[\bigoplus_{\substack{k=|m-n|\\k\equiv m+n\mod 2}}^{m+n}\OP{Sym}^k(V).\]
\end{question}

\iffalse
\begin{answer}
The representation $\OP{Sym}^n\CC^2=\bigoplus_{i=0}^n\CC\cdot e_i$ is a direct sum of weight spaces $\CC\cdot e_i$ with weight $n-2i$ and $\OP{Sym}^m\CC^2$ is a direct sum $\bigoplus_{j=0}^m\CC\cdot f_j$. Suppose that $m\geq n$ so that $|m-n|=m-n$. If we list generators by their weight, we get a diagram like this:

{\tiny
\begin{tabular}{cccccccccccc}
$e_0\otimes f_0$ & $e_0\otimes f_1$ & $e_0\otimes f_2$ & $\cdots$ & $e_0\otimes f_n$ & $e_0\otimes f_{n+1}$ & $\cdots$ & $e_0\otimes f_m$ & $e_1\otimes f_m$ & $\cdots$ & $e_n\otimes f_m$\\
& $e_1\otimes f_0$ & $e_1\otimes f_1$ & $\cdots$ & $e_1\otimes f_{n-1}$ & $e_1\otimes f_n$& $\cdots$ & $e_1\otimes f_m$ & $e_2\otimes f_{m-1}$ &  & \\
 & & $e_2\otimes f_0$ &  & $\vdots$&$\vdots$&$\vdots$&$\vdots$ & $\vdots$ & & &\\
 & & & & $\vdots$ & $\vdots$ & $\vdots$ & $\vdots$ &$e_n\otimes f_{m-n+1}$ & & &\\
&&&& $e_n\otimes f_0$ & $e_n\otimes f_1$ &$\cdots$& $e_n\otimes f_{m-n}$ & & & &
\end{tabular}
}

i.e.

\begin{center}{
\begin{tabular}{cccccccccccc}
$\bullet$ & $\bullet$ & $\cdots$ & $\bullet$ & $\bullet$ & $\cdots$ & $\bullet$ & $\bullet$ & $\cdots$ & $\bullet$ & $\bullet$\\
& $\bullet$ & $\cdots$ & $\bullet$ & $\bullet$ & $\cdots$ & $\bullet$ & $\bullet$ & $\cdots$  & $\bullet$ &\\
 & & & $\vdots$  & $\vdots$&$\vdots$&$\vdots$ & $\vdots$ & & &\\
 & & & $\bullet$ & $\vdots$ & $\vdots$ &$\vdots$ & $\bullet$ & &\\
&&& & $\bullet$ & $\cdots$& $\bullet$ & & & &
\end{tabular}
}\end{center}

The top row has length $n+m+1$ and contains a highest weight vector with weight $n+m$ so we can peel off a copy of $\OP{Sym}^{n+m}(\CC^2)$. The next row has length $n+m-1$ and contains a highest weight vector with weight $n+m-2$ so we can peel off a copy of $\OP{Sym}^{n+m-2}(\CC^2)$. Continuing in this manner we reach the last row which has length $|m-n|+1$ and we peel off a copy of $\OP{Sym}^{|m-n|}(\CC^2)$. This gives the Clebsch-Gordan formula.
\end{answer}
\newpage
\fi

\begin{question}\ \\
Prove that if $\rho\colon\mathfrak{sl}(2,\CC)\to\mk{gl}(V)$ is a representation and $X,Y,H$ denote the usual basis of $\mk{sl}(2,\CC)$ satisfying the commutation relations
\[[H,X]=2X,\quad [H,Y]=-2Y,\quad [X,Y]=H\]
then
\[C:=\rho(X)\rho(Y)+\rho(Y)\rho(X)+\frac{1}{2}\rho(H)^2\]
commutes with $\rho(X)$, $\rho(Y)$ and $\rho(H)$. Deduce that if $V=\bigoplus_{\lambda}V_{\lambda}$ is the decomposition of $V$ into eigenspaces of $C$ then each $V_{\lambda}$ is a subrepresentation. If $V$ is irreducible with highest weight $m$, deduce that $C$ is the diagonal matrix $\left(m+\frac{1}{2}m^2\right)\OP{Id}$.
\end{question}

\iffalse
\begin{answer}
In the solution we will write $X$ for $\rho(X)$, etc. and the fact that $\rho$ is a homomorphism of Lie algebras means that $[X,Y]$ means both $[\rho(X),\rho(Y)]$ and $\rho([X,Y])$ so the notation is well-defined! We have
\begin{align*}
      CX      &=(XY+YX+H^2/2)X\\
              &=XYX+YXX+HHX/2\\
              &=XYX+[Y,X]X+XYX+H[H,X]/2+HXH/2\\
              &=XYX-HX+X[Y,X]+XXY+HX+[H,X]H/2+XHH/2\\
              &=XYX-HX-XH+XXY+HX+XH+XHH/2\\
              &=X(XY+YX+H^2/2)\\
              &=XC.
\end{align*}
Similar arguments work for $Y$ and $H$.

Now since $C$ commutes with $\rho(X)$, if $Cv=\lambda v$ then
\[C\rho(X)v=\rho(X)Cv=\rho(X)\lambda v=\lambda\rho(X)v\]
so $V_{\lambda}$ is preserved by $\rho(X)$ (and $\rho(Y),\rho(H)$ by the same argument). Therefore it is a subrepresentation.

If $V$ is irreducible then $V=V_{\lambda}$ for some eigenvalue $\lambda$ and hence $Cv=\lambda v$ for all $v\in V$. To compute $\lambda$, assume that $v$ is a highest weight vector with weight $m$. Then $\rho(X)v=0$, $\rho(H)v=mv$ and $\rho(Y)v$ satisfies
\[\rho(X)\rho(Y)v=mv,\ \rho(H)\rho(Y)v=(m-2)\rho(Y)v\]
by the computations we did in the proof of the classification theorem for irreducible $\mk{sl}(2,\CC)$-representations. Therefore
\[Cv=\rho(X)\rho(Y)v+\rho(Y)\rho(X)v+\frac{1}{2}\rho(H)^2v=mv+m^2v/2\]
and $\lambda=m+m^2/2$ as required.
\end{answer}
\fi

\newpage

\bigskip
\hrule
\bigskip
If $R\colon G\to GL(V)$ is a representation we say that:
\begin{itemize}
\item $M\in V$ is {\em $R$-invariant} if $R(g)M=M$ for all $g\in G$.
\item a symmetric bilinear form $B\colon V\otimes V\to\mathbf{K}$ is {\em $R$-invariant} if
\[B(R(g)v,R(g)w)=B(v,w)\]
for all $v,w\in V$ and $g\in G$.
\end{itemize}

If $\rho\colon\mathfrak{g}\to\mathfrak{gl}(V)$ is a representation, we say that
\begin{itemize}
\item $M\in V$ is {\em $\rho$-invariant} if $\rho(X)M=0$ for all $X\in\mathfrak{g}$.
\item a symmetric bilinear form $B\colon V\times V\to\mathbf{K}$ is {\em $\rho$-invariant} if
\[B(\rho(X)v,w)+B(v,\rho(X)w)=0\]
for all $v,w\in V$ and $X\in\mathfrak{g}$.
\end{itemize}
\bigskip
\hrule
\bigskip

\begin{question}\ \\
Let $G$ be a connected Lie group and $R\colon G\to GL(V)$ be a representation. Let $\rho=R_*$ be the linearisation of $R$.

\begin{enumerate}
\item[(a)] Prove that $M\in V$ is $G$-invariant if and only if it is $R_*$-invariant.
\item[(b)] Prove that a symmetric bilinear form $B\colon V\times V\to\mathbf{K}$ is $R$-invariant if and only if it is $R_*$-invariant.
\end{enumerate}

{\em Hint: To show $M$ or $B$ is $R$-invariant it suffices to check it on an exponential chart because the group is connected and connected groups are generated by the image of an exponential chart.}

\begin{enumerate}
\item[(c)] Let $\OP{ad}\colon\mathfrak{g}\to\mathfrak{gl}(\mathfrak{g})$ be the adjoint representation $X\mapsto\OP{ad}_X$, $\OP{ad}_XY=[X,Y]$. Define the symmetric bilinear form
\[B(X,Y)=\OP{Tr}(\OP{ad}_X\OP{ad}_Y)\]
where $\OP{Tr}$ denotes the trace. Using some form of the Jacobi identity, prove that $B$ is $\OP{ad}$-invariant. This is called the {\em Killing form}.

{\em Hint: The trace of a commutator of matrices vanishes.}
\item[(d)] Let $X,H,Y$ be the usual basis for $\mathfrak{sl}(2,\CC)$. Check that with respect to this basis
\[\OP{ad}_X=\MATR{0}{-2}{0}{0}{0}{1}{0}{0}{0},\ \OP{ad}_H=\MATR{2}{0}{0}{0}{0}{0}{0}{0}{-2},\ \OP{ad}_Y=\MATR{0}{0}{0}{-1}{0}{0}{0}{2}{0}\]
and hence compute the Killing form on all pairs $B(a,b)$, $a,b\in\{X,H,Y\}$.
\end{enumerate}
\end{question}

\iffalse
\begin{answer}
\begin{enumerate}[(a)]
\item If $\rho(X)M=0$ then $\exp(\rho(X))M=(1+\rho(X)+\tfrac{1}{2}\rho(X)^2+\cdots)M=M$. But $R(\exp X)M=\exp(\rho(X))M$ so $R(\exp X)M=M$ for all $X$ and hence $R(g)M=M$ for all $g$ in a neighbourhood of 1 and hence for all $g\in G$ since $G$ is generated by a neighbourhood of 1 by connectedness of $G$.

Conversely if $R(g)M=M$ for all $g$ then $R(\exp tX)M=M$ for all $t$ and differentiating with respect to $t$ we get $\rho(X)M=0$ as required.
\item If $B$ is $R$-invariant then differentiating $B(R(e^{tX})v,R(e^{tX})w)=B(v,w)$ with respect to $t$ at $t=0$ gives
\[B(\rho(X)v,w)+B(v,\rho(X)w)=0.\]
Conversely if $B(\rho(X)v,w)=-B(v,\rho(X)w)$ then $B(\rho(X)^nv,\rho(X)^{k-n}w)=(-1)^nB(v,\rho(X)^kw)$ so
\begin{align*}
B(R(\exp X)v,R(\exp X)w)&=B(\exp(\rho(X))v,\exp(\rho(X))w)\\
&=\sum_{n,m}\frac{1}{n!m!}B(\rho(X)^nv,\rho(X)^mw)\\
&=\sum_{k=0}^{\infty}\sum_{n=0}^k\frac{1}{k!}\binom{k}{n}B(\rho(X)^nv,\rho(X)^{k-n}w)\\
&=B(v,w)+\sum_{k=1}^{\infty}\frac{1}{k!}\left(\sum_{n=0}^k\binom{k}{n}(-1)^n\right)B(v,\rho(X)^kw)\\
&=B(v,w)
\end{align*}
since all $k\geq 1$ terms contain $\sum_{n=0}^k\binom{k}{n}(-1)^n=(1-1)^k=0$.

Alternatively you can argue as follows. The $t$-derivative of $B(R(e^{tX})v,R(e^{tX})w)$ is
\[B(\rho(X)R(e^{tX})v,R(e^{tX})w)+B(R(e^{tX})v,\rho(X)R(e^{tX})w)=0\]
and certainly at $t=0$ we have $B(R(e^{0X})v,R(e^{0X})w)=B(v,w)$, hence
\[B(R(e^{tX})v,R(e^{tX})w)=B(v,w)\mbox{ for all }t.\]
This proves invariance for a neighbourhood of the identity and hence for the whole group because $G$ is connected and hence generated by a neighbourhood of the identity.
\item The form is clearly bilinear and it is symmetric because $\OP{Tr}(AB)=\OP{Tr}(BA)$. To see $\ad$-invariance we must check that
\[B(\ad_ZX,Y)+B(X,\ad_ZY)=0\]
i.e. that
\[\OP{Tr}\left(\ad_{[Z,X]}\ad_Y+\ad_X\ad_{[Z,Y]}\right)=0.\]
The Jacobi identity allows us to rewrite the expression inside the trace as
\[\ad_Z\ad_X\ad_Y-ad_X\ad_Z\ad_Y+\ad_X\ad_Z\ad_Y-\ad_X\ad_Y\ad_Z\]
or
\[[\ad_Z,\ad_X\ad_Y].\]
Since the trace of a commutator vanishes, we get $\ad$-invariance.

\item For $\mathfrak{sl}(2,\CC)$ we use the basis $H,X,Y$ to compute the Killing form. We have
\[\ad_XH=-2X,\ \ad_XY=H,\ \ad_YH=2Y,\ \ad_YX=-H,\ \ad_HX=2X,\ \ad_HY=-2Y\]
so:
\begin{itemize}
\item $\ad_H\ad_X$ sends $X$ and $Y$ to zero and $H$ to $-4X$; $\ad_H\ad_Y$ sends $X$ and $Y$ to zero and $H$ to $-4Y$. Both of these maps have trace zero.
\item $\ad_X\ad_Y$ sends $X$ to $2X$, $Y$ to $0$ and $H$ to $2H$ which has trace 4.
\item $\ad_H\ad_H$ sends $X$ to $4X$, $Y$ to $4Y$ and $H$ to zero so the trace is 8.
\item $\ad_X\ad_X$ sends $Y$ to $-2X$ and $X$ and $H$ to zero so the trace is zero.
\item $ad_Y\ad_Y$ sends $X$ to $-2Y$ and $Y$ and $H$ to zero so the trace is zero.
\end{itemize}
Therefore the Killing form with respect to this basis is
\[B(H,H)=8,\ B(X,X)=B(Y,Y)=0,\ B(X,Y)=4,\ B(H,X)=B(H,Y)=0.\]
\end{enumerate}
\end{answer}
\fi

\end{document}
