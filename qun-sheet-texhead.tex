\usepackage{parskip,palatino,amsthm,amsmath,amsfonts,amssymb}
\usepackage{multicol}
\usepackage{enumerate}
\usepackage[letterpaper,margin=1in,bottom=0.7in]{geometry}

\newcommand{\dd}[2]{\frac{d #1}{d #2}}
\newcommand{\pd}[2]{\frac{\partial #1}{\partial #2}}
\newcommand{\brf}[2]{\left(\frac{#1}{#2}\right)}
                       % Bracket-frac, e.g. for (n\pi x/L) in Fourier series
\newcommand{\fsin}[1]{\sin\brf{#1 \pi x}{L}}
\newcommand{\fcos}[1]{\cos\brf{#1 \pi x}{L}}
\newcommand{\RR}{\mathbf{R}}
\newcommand{\CC}{\mathbf{C}}
\newcommand{\ZZ}{\mathbf{Z}}
\newcommand{\mk}{\mathfrak}
\newcommand{\mb}{\mathbf}
\newcommand{\OP}{\operatorname}
\newcommand{\MATR}[9]{\left(\begin{array}{cccc}#1 & #2 & \cdots & #3\\ #4 & #5 & \cdots & #6\\ \vdots & \vdots & \ddots & \vdots\\ #7 & #8 & \cdots & #9\end{array}\right)}
\newcommand{\matr}[4]{\left(\begin{array}{cc}#1 & #2\\ #3 & #4\end{array}\right)}
\newcommand{\matt}[9]{\left(\begin{array}{ccc}#1 & #2 & #3\\#4 & #5 & #6\\#7 & #8 & #9\end{array}\right)}
\newcommand{\vect}[3]{\left(\begin{array}{c}#1 \\ #2 \\ #3\end{array}\right)}
\newcommand{\ad}{\OP{ad}}

\newtheorem{thm}{Theorem}
\newtheorem{lma}[thm]{Lemma}

\theoremstyle{definition}
\newtheorem{question}{Question}
\newtheorem{answer}{Answer}

\theoremstyle{remark}
\newtheorem*{rmk}{Remark}

%%%%%%%%%%%%%%%%%% Add extra space before theorems

\begingroup 
\makeatletter 
\@for\theoremstyle:=definition,remark,plain,TheoremNum\do{% 
\expandafter\g@addto@macro\csname th@\theoremstyle\endcsname{% 
\addtolength\thm@preskip\parskip 
}% 
} 
\endgroup 
