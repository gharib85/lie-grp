\documentclass[12pt]{article}

\usepackage{parskip,palatino,amsthm,amsmath,amsfonts,amssymb}
\usepackage{multicol}
\usepackage{enumerate}
\usepackage[letterpaper,margin=1in,bottom=0.7in]{geometry}

\newcommand{\dd}[2]{\frac{d #1}{d #2}}
\newcommand{\pd}[2]{\frac{\partial #1}{\partial #2}}
\newcommand{\brf}[2]{\left(\frac{#1}{#2}\right)}
                       % Bracket-frac, e.g. for (n\pi x/L) in Fourier series
\newcommand{\fsin}[1]{\sin\brf{#1 \pi x}{L}}
\newcommand{\fcos}[1]{\cos\brf{#1 \pi x}{L}}
\newcommand{\RR}{\mathbf{R}}
\newcommand{\CC}{\mathbf{C}}
\newcommand{\ZZ}{\mathbf{Z}}
\newcommand{\mk}{\mathfrak}
\newcommand{\mb}{\mathbf}
\newcommand{\OP}{\operatorname}
\newcommand{\MATR}[9]{\left(\begin{array}{cccc}#1 & #2 & \cdots & #3\\ #4 & #5 & \cdots & #6\\ \vdots & \vdots & \ddots & \vdots\\ #7 & #8 & \cdots & #9\end{array}\right)}
\newcommand{\matr}[4]{\left(\begin{array}{cc}#1 & #2\\ #3 & #4\end{array}\right)}
\newcommand{\matt}[9]{\left(\begin{array}{ccc}#1 & #2 & #3\\#4 & #5 & #6\\#7 & #8 & #9\end{array}\right)}
\newcommand{\vect}[3]{\left(\begin{array}{c}#1 \\ #2 \\ #3\end{array}\right)}
\newcommand{\ad}{\OP{ad}}

\newtheorem{thm}{Theorem}
\newtheorem{lma}[thm]{Lemma}

\theoremstyle{definition}
\newtheorem{question}{Question}
\newtheorem{answer}{Answer}

\theoremstyle{remark}
\newtheorem*{rmk}{Remark}

%%%%%%%%%%%%%%%%%% Add extra space before theorems

\begingroup 
\makeatletter 
\@for\theoremstyle:=definition,remark,plain,TheoremNum\do{% 
\expandafter\g@addto@macro\csname th@\theoremstyle\endcsname{% 
\addtolength\thm@preskip\parskip 
}% 
} 
\endgroup 

\include{diagrams}

\title{Sheet 2: Matrix groups}
\author{J. Evans}
\date{}

\begin{document}
\maketitle


\begin{question}\ \\
Prove that $A^T=-A$ if and only if $\exp(tA)\in O(n)$ for all $t\in\RR$. This implies that the Lie algebra of $O(n)$ is the space $\mathfrak{o}(n)$ of antisymmetric matrices.
\end{question}

\begin{answer}
If $A^T=-A$ then $(\exp(tA))^T=\left(\sum_{n=0}^{\infty}\frac{1}{n!}A^n\right)^T=\sum_{n=0}^{\infty}\frac{1}{n!}(A^T)^n=\sum_{n=0}^{\infty}\frac{1}{n!}(-1)^nA^n=\exp(-tA)$ hence $\exp(tA)\left(\exp(tA)\right)^T=1$ and $\exp(tA)\in O(n)$.

Conversely if $\exp(tA)\left(\exp(tA)\right)^T=1$ for all $t$ then differentiating with respect to $t$ at $t=0$ we get
\[A+A^T=0\]
and hence $A^T=-A$.
\end{answer}
\newpage

\bigskip

\begin{question}
Show that the third order term in the Baker-Campbell-Hausdorff formula for $\log(\exp A\exp B)$ is
\[\frac{1}{12}[A,[A,B]]-\frac{1}{12}[B,[A,B]].\]
{\em If you think this is getting messy, it just means you're on the right track. Everything will simplify beautifully.}
\end{question}

\begin{answer}
Let
\[e^Ae^B-1=X=\left(1+A+A^2/2+A^3/3!+\cdots\right)\left(1+B+B^2/2+B^3/3!+\cdots\right)-1\]
and compute
\[\log(e^Ae^B)=\log(1+X)=X-X^2/2+X^3/3+\cdots\]
to third order. Thereafter, good luck. Be careful to note that $A$ and $B$ do not commute.
\end{answer}
\newpage

\bigskip

\begin{question}\ \\
\begin{enumerate}
\item[(a)] Suppose that $W_1$ and $W_2$ are complementary subspaces of $\mathfrak{gl}(n,\RR)$, so that $\mathfrak{gl}(n,\RR)=W_1\oplus W_2$. Consider the map
\[F\colon W_1\oplus W_2\to GL(n,\RR),\quad F(w_1\oplus w_2)=\exp(w_1)\exp(w_2).\]
By computing the Taylor expansion of $F$ (plugging in the Taylor expansions of $\exp(w_1)$ and $\exp(w_2)$ and expanding), show that $d_{(0,0)}F(w_1\oplus w_2)=w_1+w_2$.
\item[(b)] Deduce that there are open neighbourhoods $0\in U'\subset\mathfrak{gl}(n,\RR)$ and $1\in V'\subset GL(n,\RR)$ such that $F|_{U'}\colon U'\to V'$ is a diffeomorphism.
\item[(c)] Let $G\subset GL(n,\RR)$ be a matrix group, set $W_1=\mathfrak{g}$ and let $W_2$ be a complement for $V$. Given $g\in G$, define
\[F_g\colon\mathfrak{g}\oplus W_2\to GL(n,\RR),\quad F_g(w_1\oplus w_2)=g\exp(w_1)\exp(w_2)\]
Show that there are open sets $0\in U'\subset\mathfrak{gl}(n,\RR)$ and $g\in V'\subset GL(n,\RR)$ such that $F_g|_{U'}$ is a diffeomorphism.
\end{enumerate}
\end{question}

\begin{answer}
\begin{enumerate}[(a)]
\item We have $F(w_1\oplus w_2)=(1+w_1+\cdots)(1+w_2+\cdots)=1+w_1+w_2+\cdots$ so the first order term in the Taylor expansion is $w_1+w_2$ (i.e. the differential $d_{(0,0)}F(w_1\oplus w_2)=w_1+w_2$ is the identity).
\item Since the differential is the identity and in particular invertible, the inverse function theorem guarantees the existence of a local smooth inverse for $F$ restricted to a sufficiently small neighbourhood of $0\in W_1\oplus W_2$.
\item In this case the differential is $d_{(0,0)}F_g(w_1\oplus w_2)=gw_1+gw_2$ in other words it is multiplication by $g\in GL(n,\RR)$ which is invertible so the inverse function theorem guarantees that $F_g$ is a local diffeomorphism. This provides exponential charts near every point $g\in G$.
\end{enumerate}
\end{answer}
\newpage

\bigskip
\hrule
\bigskip
The last two questions study the following formula and its applications.

\begin{lma}[Jacobi formula\footnote{Not to be confused with the Jacobi identity later in the course.}]
Let $A(t)$ be a path of invertible matrices ($t\in\RR$). Then
\begin{equation}\label{eqn:jacformula}
\frac{d}{dt}\det(A(t))=\det(A(t))\OP{Tr}(A^{-1}\dot{A}(t)).
\end{equation}
Here $\OP{Tr}(M)$ denotes the trace of $M$ (the sum of its diagonal entries) and $\dot{A}(t)$ denotes the matrix whose entries are the $t$-derivatives of the entries of $A$.
\end{lma}
\bigskip
\hrule
\bigskip

\begin{question}(Proof of Jacobi formula)\\
\begin{enumerate}
\item[(a)] Show that $\det(1+\epsilon H)=1+\epsilon\OP{Tr}(H)+\mathcal{O}(\epsilon^2)$. {\em Hint: It might help to write out}
\[\det(1+\epsilon H)=\det\MATR{1+\epsilon H_{11}}{\epsilon H_{12}}{\epsilon H_{1n}}{\epsilon H_{21}}{1+\epsilon H_{22}}{\epsilon H_{2n}}{\epsilon H_{n1}}{\epsilon H_{n2}}{1+\epsilon H_{nn}}.\]
\item[(b)] Show that $\det(A+\epsilon H)=\det(A)+\epsilon\det(A)\OP{Tr}(A^{-1}H)+\mathcal{O}(\epsilon^2)$.
\item[(c)] Suppose that $A(t+\epsilon)=A(t)+\epsilon\dot{A}(t)+\mathcal{O}(\epsilon^2)$ is the Taylor expansion of a path $A(t)$ of matrices. Deduce Jacobi's formula \eqref{eqn:jacformula}.
\end{enumerate}
\end{question}

\begin{answer}
\begin{enumerate}[(a)]
\item We have
\begin{align*}
\det(1+\epsilon H)&=\det\MATR{1+\epsilon H_{11}}{\epsilon H_{12}}{\epsilon H_{1n}}{\epsilon H_{21}}{1+\epsilon H_{22}}{\epsilon H_{2n}}{\epsilon H_{n1}}{\epsilon H_{n2}}{1+\epsilon H_{nn}}\\
&=(1+\epsilon H_{11})(1+\epsilon H_{22})\cdots(1+\epsilon H_{nn})+\mathcal{O}(\epsilon^2)
\end{align*}
where the product comes from the diagonal terms and all other products contributing to the determinant involve at least two off-diagonal terms and hence have a factor of $\epsilon^2$. This expands to
\[\det(1+\epsilon H)=1+\epsilon\sum H_{ii}+\mathcal{O}(\epsilon^2)\]
and $\OP{Tr}(H)=\sum H_{ii}$.
\item We have $\det(A+\epsilon H)=\det(A(1+\epsilon A^{-1}H))=\det(A)\det(1+\epsilon A^{-1}H)$ so $\det(A+\epsilon H)=\det(A)+\epsilon\det(A)\OP{Tr}(A^{-1}H)+\mathcal{O}(\epsilon^2)$ by part (a).
\item If $A(t)$ is a path of matrices then $\det(A(t+\epsilon))=\det(A(t)+\epsilon\dot{A}(t)+\mathcal{O}(\epsilon^2))=\det(A(t))+\epsilon\det(A)\OP{Tr}(A^{-1}(t)\dot{A}(t))+\mathcal{O}(\epsilon^2)$. Since the derivative is just the first order part of the Taylor series we have Jacobi's formula
\[\frac{d}{dt}\det(A(t))=\det(A)\OP{Tr}(A^{-1}(t)\dot{A}(t)).\]
\end{enumerate}
\end{answer}
\newpage

\bigskip

\begin{question}(Application of Jacobi formula)\\
Given an $n$-by-$n$ matrix $H$, let $\phi(t)=\det\exp(tH)$.
\begin{enumerate}
\item[(a)] Deduce from the Jacobi formula that $\dot{\phi}(t)=\phi(t)\OP{Tr}(H)$.
\item[(b)] Deduce from (a) that $\det(\exp H)=\exp\OP{Tr}(H)$.
\end{enumerate}
Let $SL(n,\RR)$ denote the group of $n$-by-$n$ matrices with determinant one and $\mathfrak{sl}(n,\RR)$ denote the space of $n$-by-$n$ matrices with trace zero.
\begin{enumerate}
\item[(c)] Deduce that $H\in\mathfrak{sl}(n,\RR)$ if and only if $\exp(tH)\in SL(n,\RR)$ for all $t\in\RR$. {\em This implies that $\mathfrak{sl}(n,\RR)$ is the Lie algebra of $SL(n,\RR)$.}
\end{enumerate}
\end{question}

\begin{answer}
\begin{enumerate}[(a)]
\item We have $\frac{d}{dt}\exp(tH)=\exp(tH)H$ so the Jacobi formula gives $\frac{d}{dt}\det(\exp(tH))=\det(\exp(tH))\OP{Tr}(\exp(-tH)\exp(tH)H)$ and if $\phi(t)=\det(\exp(tH))$ then $\dot{\phi}(t)=\phi(t)\OP{Tr}(H)$.
\item We also know that $\phi(0)=1$ so this tells us that $\phi(t)$ is a solution to the ODE $\dot{\phi}(t)=\phi(t)\OP{Tr}(H)$ with $\phi(0)=1$. Another solution is $\exp(t\OP{Tr}(H))$. By uniqueness of solutions to ODEs with given initial conditions we get that $\phi(t)=\exp(t\OP{Tr}(H))$.
\item If $H\in\mathfrak{sl}(n,\RR)$ then $\OP{Tr}(H)=0$ so $\exp(t\OP{Tr}(H))=0$ so $\det(\exp(tH))=\exp(0)=1$ so $\exp(tH)\in SL(n,\RR)$. Conversely if $\exp(tH)\in SL(n,\RR)$ then $\det(\exp(tH))=1$ for all $t$ so differentiating with respect to $t$ at $t=0$ we get $\OP{Tr}(H)=0$.
\end{enumerate}
\end{answer}

\end{document}
