\documentclass[12pt]{article}

\usepackage{parskip,palatino,amsthm,amsmath,amsfonts,amssymb}
\usepackage{multicol}
\usepackage{enumerate}
\usepackage[letterpaper,margin=1in,bottom=0.7in]{geometry}

\newcommand{\dd}[2]{\frac{d #1}{d #2}}
\newcommand{\pd}[2]{\frac{\partial #1}{\partial #2}}
\newcommand{\brf}[2]{\left(\frac{#1}{#2}\right)}
                       % Bracket-frac, e.g. for (n\pi x/L) in Fourier series
\newcommand{\fsin}[1]{\sin\brf{#1 \pi x}{L}}
\newcommand{\fcos}[1]{\cos\brf{#1 \pi x}{L}}
\newcommand{\RR}{\mathbf{R}}
\newcommand{\CC}{\mathbf{C}}
\newcommand{\ZZ}{\mathbf{Z}}
\newcommand{\mk}{\mathfrak}
\newcommand{\mb}{\mathbf}
\newcommand{\OP}{\operatorname}
\newcommand{\MATR}[9]{\left(\begin{array}{cccc}#1 & #2 & \cdots & #3\\ #4 & #5 & \cdots & #6\\ \vdots & \vdots & \ddots & \vdots\\ #7 & #8 & \cdots & #9\end{array}\right)}
\newcommand{\matr}[4]{\left(\begin{array}{cc}#1 & #2\\ #3 & #4\end{array}\right)}
\newcommand{\matt}[9]{\left(\begin{array}{ccc}#1 & #2 & #3\\#4 & #5 & #6\\#7 & #8 & #9\end{array}\right)}
\newcommand{\vect}[3]{\left(\begin{array}{c}#1 \\ #2 \\ #3\end{array}\right)}
\newcommand{\ad}{\OP{ad}}

\newtheorem{thm}{Theorem}
\newtheorem{lma}[thm]{Lemma}

\theoremstyle{definition}
\newtheorem{question}{Question}
\newtheorem{answer}{Answer}

\theoremstyle{remark}
\newtheorem*{rmk}{Remark}

%%%%%%%%%%%%%%%%%% Add extra space before theorems

\begingroup 
\makeatletter 
\@for\theoremstyle:=definition,remark,plain,TheoremNum\do{% 
\expandafter\g@addto@macro\csname th@\theoremstyle\endcsname{% 
\addtolength\thm@preskip\parskip 
}% 
} 
\endgroup 

\include{diagrams}

\title{Sheet 5: Representations of Lie groups}
\author{J. Evans}
\date{}

\begin{document}
\maketitle

\begin{question}\ \\
Let $x,y$ be a basis for the standard representation $\CC^2$ of $SU(2)$ and let $x\otimes x$, $\frac{1}{2}(x\otimes y+y\otimes x)$, $y\otimes y$ be a basis for $\OP{Sym}^2\CC^2$. Write down a matrix for the action of $\matr{\alpha}{\beta}{-\bar{\beta}}{\bar{\alpha}}\in SU(2)$ on $\OP{Sym}^2\CC^2$ in terms of this basis.
\end{question}

\iffalse
\begin{answer}
We have
\begin{align*}
\OP{Sym}^2\matr{\alpha}{\beta}{-\bar{\beta}}{\bar{\alpha}}(x\otimes x)&=(\alpha x-\bar{\beta} y)\otimes(\alpha x-\bar{\beta} y)\\
&=\alpha^2(x\otimes x)-2\alpha\bar{\beta}\left(\frac{x\otimes y+y\otimes x}{2}\right)+\bar{\beta}^2y\otimes y\\
\OP{Sym}^2\matr{\alpha}{\beta}{-\bar{\beta}}{\bar{\alpha}}(y\otimes y)&=(\beta x+\bar{\alpha} y)\otimes(\beta x+\bar{\alpha} y)\\
&=\beta^2x\otimes x+2\bar{\alpha}\beta\left(\frac{x\otimes y+y\otimes x}{2}\right)+\bar{\alpha}^2y\otimes y
\end{align*}
and
\begin{align*}
\OP{Sym}^2\matr{\alpha}{\beta}{-\bar{\beta}}{\bar{\alpha}}\left(\frac{x\otimes y+y\otimes x}{2}\right)&=\frac{1}{2}(\alpha x-\bar{\beta} y)\otimes(\beta x+\bar{\alpha}y)+\frac{1}{2}({\beta}x+\bar{\alpha}y)\otimes(\alpha x-\bar{\beta} y)\\
&=\alpha{\beta}x\otimes x+(|\alpha|^2-|\beta|^2)\left(\frac{x\otimes y+y\otimes x}{2}\right)-\bar{\alpha}\bar{\beta} y\otimes y
\end{align*}
so the matrix is
\[\matt{\alpha^2}{\alpha{\beta}}{{\beta}^2}{-2\alpha\bar{\beta}}{|\alpha|^2-|\beta|^2}{2\bar{\alpha}\beta}{\bar{\beta}^2}{-\bar{\alpha}\bar{\beta}}{\bar{\alpha}^2}.\]
\end{answer}
\newpage
\fi


\bigskip


\begin{question}\ \\
\begin{enumerate}
\item[(a)] Suppose that $R\colon G\to GL(V)$ is a representation. Show that $R^*\colon G\to GL(V^*)$ is a representation.
\item[(b)] Suppose that $R_1\colon G\to GL(V_1)$ and $R_2\colon G\to GL(V_2)$ are representations and that $L\colon V_1\to V_2$ is a morphism of representations. Show that the kernel and image of $L$ are subrepresentations of $V_1$ and $V_2$ respectively. If $G$ is compact and $L$ is surjective, deduce that $V_1\cong \ker L\oplus V_2$ as representations.
\item[(c)] Prove that the trace map $\OP{Tr}\colon\mathfrak{gl}(n,\RR)\to\RR$ is a morphism from the adjoint representation of $GL(n,\RR)$ to the trivial one-dimensional representation. Deduce that $\mathfrak{sl}(n,\RR)$ is a subrepresentation.
\item[(d)] Let $V$ denote the standard representation of $SU(2)$. Check that the map
\begin{align*}
\OP{Sym}^pV\otimes\OP{Sym}^qV&\to\OP{Sym}^{p+q}V,\\
x_1\cdots x_p\otimes x_{p+1}\cdots x_{p+q}&\mapsto x_1\cdots x_px_{p+1}\cdots x_{p+q}
\end{align*}
is a morphism of $SU(2)$-representations. In the case that $p=1$, $q=2$:
\begin{enumerate}
\item[(i)] find the kernel;
\item[(ii)] show that the kernel is isomorphic to $V$;
\item[(iii)] deduce that $V\otimes\OP{Sym}^2V\cong V\oplus\OP{Sym}^3V$.
\end{enumerate}
\end{enumerate}
\end{question}

\iffalse
\begin{answer}
\begin{enumerate}[(a)]
\item We need to check that $R^*(gh)=R^*(g)R^*(h)$ and $R^*(1)=1$. For $f\in V^*$ and $v\in V$ we have
\begin{align*}
(R^*(gh)f)(v)&=f((gh)^{-1}v)\\
             &=f(h^{-1}g^{-1}v)\\
             &=(R^*(h)f)(g^{-1}v)\\
             &=(R^*(g)R^*(h)f)(v)\\
(R^*(1)f)(v) &=f(v)
\end{align*}
as required.
\item Let $F$ be a morphism of representations and consider the subspace $\ker F=\{v\ :\ F(v)=0\}$. We must show that this is preserved by $R_1(g)$ for each $g$. We have $F(R_1(g)v)=R_2(F(v))=0$ if $F(v)=0$ so $R_1(g)$ preserves $\ker F$. Similarly, if $w\in\OP{im} F$ then $R_2(g)w=R_2(g)F(v)$ for some $v$ and so
\[R_2(g)w=R_2(g)F(v)=F(R_1(g)v)\in\OP{im} F\]
since $F$ is a morphism of representations.

Consider an invariant Hermitian inner product on $V_1$ (possible because $G$ is compact) and let $W\subset V_1$ be the orthogonal complement for $\ker F$ so that $W$ is a subrepresentation and $V_1\cong W\oplus \ker F$. The restriction $F|_W\colon W\to \OP{im} F$ is now an isomorphism of representations so if $F$ is surjective we get $V_1\cong V_2\oplus\ker F$.
\item The map $\OP{Tr}$ is a morphism if $\OP{Tr}(\OP{Ad}_gv)=\OP{Tr}(v)$ for all $g\in GL(n,\CC)$ and $v\in\mathfrak{gl}(n,\CC)$. Since $\OP{Ad}_gv=gvg^{-1}$ we have
\[\OP{Tr}(\OP{Ad}_gv)=\OP{Tr}(gvg^{-1})=\OP{Tr}(v)\]
so $\OP{Tr}$ is a morphism. In particular, by the previous part of the question, its kernel (the subspace of tracefree matrices) is a subrepresentation.
\item One can check that this map is a morphism by applying $R^{\otimes n}(g)$ to each side and checking they agree. In the case that $p=1$ and $q=2$:
\begin{enumerate}
\item[(i)] The effect of this map on a basis is $x\otimes x^2\mapsto x^3$, $x\otimes xy,y\otimes x^2\mapsto x^2y$, $x\otimes y^2,y\otimes xy\mapsto xy^2$, $y\otimes y^2\mapsto y^3$. Therefore the kernel is generated by $x\otimes xy-y\otimes x^2$ and $y\otimes xy-x\otimes y^2$.
\item[(ii)] The kernel is a 2-dimensional representation. If we call this representation $\rho$ then we get (after some work)
\[\rho\matr{\alpha}{\beta}{-\bar{\beta}}{\bar{\alpha}}(x\otimes xy-y\otimes x^2)=\alpha(x\otimes xy-y\otimes x^2)+\beta(x\otimes y^2-y\otimes xy)\]
and
\[\rho\matr{\alpha}{\beta}{-\bar{\beta}}{\bar{\alpha}}(x\otimes y^2-y\otimes xy)=-\bar{\beta}(x\otimes xy-y\otimes x^2)+\bar{\alpha}(x\otimes y^2-y\otimes xy).\]
Therefore the kernel is isomorphic to the standard representation.
\item[(iii)] Part (b) tells us that $V\otimes \OP{Sym}^2V$ is isomorphic to $\ker F\oplus \OP{Sym}^3V$ since $F$ is certainly surjective. Part (d.ii) tells us that $\ker F\cong V$.
\end{enumerate}
\end{enumerate}
\end{answer}
\fi
\newpage

\begin{question}\ \\
\begin{enumerate}
\item[(a)] Suppose that $\mathbf{K}$ is a field of characteristic zero and $V=\mathbf{K}\langle e_1,e_2\rangle$. Show that
\[V^{\otimes 2}=\OP{Sym}^2(V)\oplus\Lambda^2(V).\]
\item[(b)] Let $e_i$ be a basis of $\RR^n$ and, for $i=1,\ldots,n$, let $a_i$ be the vector $\sum_{i=1}^n a_i^je_j$. By inspecting the formula for the alternating map, show that $a_1\wedge\cdots\wedge a_n=\det(a^j_i)(e_1\wedge\cdots\wedge e_n)$.
\end{enumerate}
\end{question}

\iffalse
\begin{answer}
\begin{enumerate}
\item[(a)]
We have $V^{\otimes 2}=\mathbf{K}\langle e_1\otimes e_1,e_1\otimes e_2,e_2\otimes e_1,e_2\otimes e_2\rangle$. The symmetric square is a 3-dimensional subspace containing $e_1\otimes e_1$, $e_2\otimes e_2$ and $e_1\otimes e_2+e_2\otimes e_1$. A complementary subspace is spanned by $e_1\otimes e_2-e_2\otimes e_1$ which also spans $\Lambda^2V$.

\item[(b)] Let $A$ denote the square matrix with entries $a_i^j$. We have
\begin{align*}
a_1\wedge\cdots\wedge a_n&=\frac{1}{n!}\sum_{\sigma\in S_n}(-1)^{|\sigma|}a_{\sigma(1)}\otimes\cdots\otimes a_{\sigma(n)}\\
&=\frac{1}{n!}\sum_{\sigma}(-1)^{|\sigma|}\sum_{i_1,\ldots,i_n}a_{\sigma(1)}^{i_1}e_{i_1}\otimes\cdots\otimes a_{\sigma(n)}^{i_n}e_{i_n}
\end{align*}
We ask: what is the coefficient of $e_{i_1}\otimes\cdots\otimes e_{i_n}$? Clearly it is
\[\frac{1}{n!}\sum_{\sigma}(-1)^{|\sigma|}a^{i_1}_{\sigma(1)}\cdots a^{i_n}_{\sigma(n)}\]
which is $1/n!$ times the formula for the determinant of the matrix $B$ whose entries are $b^k_{\ell}=a^{i_k}_{\sigma(\ell)}$. This is nonzero only if $i_1,\ldots,i_n$ is a permutation of $1,\ldots,n$ for otherwise two of the columns agree. Therefore $B$ is the result of permuting the rows and columns of $A$ and (remembering that a transposition of rows or columns changes the sign of the determinant) we get
\[a_1\wedge\cdots\wedge a_n=\sum_{\tau\in S_n}(-1)^{|\tau|}(1/n!)\det(A)e_{\tau(1)}\otimes\cdots\otimes e_{\tau(n)}=\det(A)e_1\wedge\cdots\wedge e_n.\]
\end{enumerate}
\end{answer}
\newpage
\fi

\bigskip

\begin{question}\ \\
Let $V=\mathbf{K}\langle e_1,e_2\rangle$ and $W=\mathbf{K}\langle f_1,f_2\rangle$ be vector spaces and consider an element $t=\sum_{i,j=1}^2t_{ij}e_i\otimes e_j\in V\otimes W$ in their tensor product. Show that there exist $v\in V$ and $w\in W$ such that $t=v\otimes w$ (i.e. $t$ is a pure tensor) if and only if $t_{11}t_{22}=t_{12}t_{21}$.

{\em This equation is called the Pl\"{u}cker relation. For tensor products of higher dimensional vector spaces there are many more Pl\"{u}cker relations $t_{ij}t_{k\ell}=t_{i\ell}t_{kj}$. This tells us that the pure tensors form a subvariety of $V\otimes W$ cut out by a collection of homogeneous polynomials of degree 2.}
\end{question}

\iffalse
\begin{answer}
If $\sum t_{ij}e_i\otimes e_j$ is a pure tensor then it is equal to $\sum v_ie_i \otimes \sum w_je_j$ and hence $t_{ij}=v_iw_j$. This means $t_{11}t_{22}=v_1w_1v_2w_2=v_1w_2v_2w_1=t_{12}t_{21}$.

Conversely, suppose the Pl\"{u}cker relation holds. If all $t_{ij}=0$ then just take $v=w=0$. If one of the $t_{ij}\neq 0$ then (renumbering the basis) we can assume it is $t_{11}$. Then set $v_1=1$, $w_1=t_{11}$, $v_2=t_{21}/t_{11}$ and $w_2=t_{12}$. We get $v\otimes w=\sum t_{ij}e_i\otimes e_j$.
\end{answer}
\newpage
\fi

\bigskip

\begin{question}\ \\
Suppose that $R\colon G\to GL(V)$ and $S\colon G\to GL(W)$ are representations and denote by $T$ the representation of $G$ on $\OP{Hom}(V,W)$ defined by
\[(T(g)F)(v)=S(g)F(R(g^{-1})v).\]
\begin{enumerate}
\item[(a)] Show that a vector $F\in\OP{Hom}(V,W)$ satisfies $T(g)F=F$ for all $g\in G$ if and only if $F$ is a morphism of representations. If $V$ and $W$ are irreducible, show that either $F=0$ or $F$ is an isomorphism.
\item[(b)] Define a map $\Phi\colon V^*\otimes W\to\OP{Hom}(V,W)$ by defining it on pure tensors as
\[\Phi(f\otimes w)(v)=f(v)w\]
and extending linearly. Check that this is an isomorphism of representations $R^*\otimes S\cong T$.

{\em Hint: Once you know it's a morphism of representations, to show it's an isomorphism, pick bases $e_i$ of $V$ and $f_j$ of $W$ and check that $\Phi(e_i^*\otimes f_j)$ is a basis for $\OP{Hom}(V,W)$. Why does this tell you it is an isomorphism?}
\end{enumerate}
\end{question}

\iffalse
\begin{answer}
\begin{enumerate}[(a)]
\item If $T(g)F=F$ for all $g\in G$ then $S(g)F(R(g^{-1})v)=F(v)$ and hence $F(R(h)v)=S(h)(F(v))$ where $h=g^{-1}$ runs over all of $G$. Therefore $F$ is a morphism of representations. If $V$ and $W$ are irreducible then the kernel and image are subrepresentations of $V$ and $W$ respectively and so either $\ker F=0$, $\OP{im} F=W$ or $\ker F=V$, $\OP{im} F=0$. In the first case $F$ is an isomorphism, in the second it is zero.
\item To check it's a morphism of representations we need to show that
\[\Phi(R^*(g)f\otimes S(g)w)v=(T(g)\Phi(f\otimes w))(v)\]
The LHS is
\[f(R(g^{-1})v)S(g)w\]
while the RHS is
\[S(g)\Phi(f\otimes w)(R(g^{-1})v)=S(g)[f(R(g^{-1})v)w].\]
Since $f(R(g^{-1})v)$ is just a scalar, this equals $f(R(g^{-1})v)S(g)w$.

To see that it's an isomorphism we observe that $e_i^*\otimes f_j$ goes to the matrix with a one in the $i$th column and $j$th row. $\Phi$ therefore surjects onto the space of matrices as $i$ and $j$. Since $\dim(V^*\otimes W)=\dim V\times \dim W=\dim\OP{Hom}(V,W)$ it is also an injection and hence an isomorphism.
\end{enumerate}
\end{answer}
\fi


\end{document}