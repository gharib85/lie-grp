\documentclass[12pt]{article}

\usepackage{parskip,palatino,amsthm,amsmath,amsfonts,amssymb}
\usepackage{multicol}
\usepackage{enumerate}
\usepackage[letterpaper,margin=1in,bottom=0.7in]{geometry}

\newcommand{\dd}[2]{\frac{d #1}{d #2}}
\newcommand{\pd}[2]{\frac{\partial #1}{\partial #2}}
\newcommand{\brf}[2]{\left(\frac{#1}{#2}\right)}
                       % Bracket-frac, e.g. for (n\pi x/L) in Fourier series
\newcommand{\fsin}[1]{\sin\brf{#1 \pi x}{L}}
\newcommand{\fcos}[1]{\cos\brf{#1 \pi x}{L}}
\newcommand{\RR}{\mathbf{R}}
\newcommand{\CC}{\mathbf{C}}
\newcommand{\ZZ}{\mathbf{Z}}
\newcommand{\mk}{\mathfrak}
\newcommand{\mb}{\mathbf}
\newcommand{\OP}{\operatorname}
\newcommand{\MATR}[9]{\left(\begin{array}{cccc}#1 & #2 & \cdots & #3\\ #4 & #5 & \cdots & #6\\ \vdots & \vdots & \ddots & \vdots\\ #7 & #8 & \cdots & #9\end{array}\right)}
\newcommand{\matr}[4]{\left(\begin{array}{cc}#1 & #2\\ #3 & #4\end{array}\right)}
\newcommand{\matt}[9]{\left(\begin{array}{ccc}#1 & #2 & #3\\#4 & #5 & #6\\#7 & #8 & #9\end{array}\right)}
\newcommand{\vect}[3]{\left(\begin{array}{c}#1 \\ #2 \\ #3\end{array}\right)}
\newcommand{\ad}{\OP{ad}}

\newtheorem{thm}{Theorem}
\newtheorem{lma}[thm]{Lemma}

\theoremstyle{definition}
\newtheorem{question}{Question}
\newtheorem{answer}{Answer}

\theoremstyle{remark}
\newtheorem*{rmk}{Remark}

%%%%%%%%%%%%%%%%%% Add extra space before theorems

\begingroup 
\makeatletter 
\@for\theoremstyle:=definition,remark,plain,TheoremNum\do{% 
\expandafter\g@addto@macro\csname th@\theoremstyle\endcsname{% 
\addtolength\thm@preskip\parskip 
}% 
} 
\endgroup 

\include{diagrams}

\title{Sheet 4: Lie's theorem}
\author{J. Evans}
\date{}

\begin{document}
\maketitle

Recall Lie's theorem:
\begin{thm}[Lie]
Let $G$ and $H$ be two path-connected matrix groups with Lie algebras $\mathfrak{g}$ and $\mathfrak{h}$. Suppose moreover that $G$ is simply-connected. For any Lie algebra homomorphism $f\colon\mathfrak{g}\to\mathfrak{h}$ there is a unique smooth homomorphism $F\colon G\to H$ with $F_*=f$.
\end{thm}

\begin{question}
Suppose that $G$ and $H$ are both path-connected, simply-connected matrix groups and that $\mathfrak{g}\cong\mathfrak{h}$ as Lie algebras. Prove that $G\cong H$. (You may assume Lie's theorem).
\end{question}

\iffalse
\begin{answer}
Let $f\colon\mathfrak{g}\to\mathfrak{h}$ be an isomorphism of the Lie algebras. Since $G$ is connected and simply-connected and $H$ is connected, this exponentiates (by Lie's theorem) to a homomorphism $F\colon G\to H$. The inverse $f^{-1}$ also exponentiates to $\tilde{F}\colon H\to G$ since $H$ is also simply-connected. The composition $\tilde{F}\circ F$ has differential $f^{-1}\circ f=\OP{id}$ and by the uniqueness part of Lie's theorem it must itself be the identity. Therefore $\tilde{F}=F^{-1}$ and $F$ is an isomorphism.
\end{answer}
\fi


The remaining exercises form a guided proof of Lie's theorem. The first ingredient we will need is a compatibility condition for a system of partial differential equations to have a solution.

\begin{question}[Maurer-Cartan equation]
In this question, you can freely assume the existence\footnote{All of the ODEs we come across will have solutions for all time.} and uniqueness of solutions to {\em ordinary} differential equations. Let $G$ be a matrix group with Lie algebra $\mathfrak{g}$. Let $\xi(s,t)$ and $\eta(s,t)$ be two smooth maps $[0,1]\times[0,1]\to\mathfrak{g}$. Prove that the following are equivalent:
\begin{enumerate}
\item[(a)] The {\em partial} differential equations
\[(\star)\quad\pd{\phi}{s}(s,t)=\phi(s,t)\xi(s,t),\quad(\star\star)\quad\pd{\phi}{t}(s,t)=\phi(s,t)\eta(s,t)\]
have a solution $\phi\colon [0,1]\times[0,1]\to G$ with $\phi(0,0)=1$.
\item[(b)] The {\em Maurer-Cartan equation} holds:
\[(\star\star\star)\quad\pd{\xi}{t}-\pd{\eta}{s}=[\xi,\eta].\]
\end{enumerate}
{\em Hints:

For (a) implies (b), try cross-differentiating $(\star)$ and $(\star\star)$ and seeing what turns up.

For (b) implies (a): This is tricky. Construct $\phi(s,0)$ first by considering $(\star)$ as an ODE at $t=0$. Then use $\phi(s,0)$ as the initial conditions for $(\star\star)$, considered as an ODE for each fixed value of $s$. By construction this solves $(\star\star)$ and it solves $(\star)$ along $t=0$; it remains to show that $(\star)$ holds everywhere. For fixed $s$, consider $\beta(t)=\pd{\phi}{s}(s,t)-\phi(s,t)\xi(s,t)$ and show (using $(\star\star)$ and $(\star\star\star)$) that $\dd{\beta}{t}=\beta\eta$; deduce that $\beta\equiv 0$ using uniqueness of solutions to ODEs and the initial condition $(\star)$ along $t=0$.}
\end{question}

\iffalse
\begin{answer}
{\bf (a) implies (b):} Given such a solution we differentiate to get
\[\frac{\partial^2\phi}{\partial s\partial t}=\frac{\partial\phi}{\partial t}\xi+\phi\frac{\partial\xi}{\partial t}=\frac{\partial\phi}{\partial s}\eta+\phi\frac{\partial\eta}{\partial s}\]
so using ($\star$) we get
\[\phi(\partial_t\xi-\partial_s\eta)=\phi[\xi,\eta]\]
which gives (b).

{\bf (b) implies (a):} We first solve the ODE
\[\frac{d\phi}{ds}(s,0)=\phi(s,0)\xi(s,0)\]
when $t=0$. Now we use $\phi(s,0)$ as an initial condition for the ODE
\[\frac{d\phi}{dt}(s,t)=\phi(s,t)\eta(s,t)\]
for each fixed $s$. This allows us to define $\phi(s,t)$ for all $(s,t)$ and by construction it satisfies $\partial\phi/\partial t=\phi\eta$. It only remains to check that $\partial\phi/\partial s=\phi\xi$. Define $\beta(s,t)=\frac{\partial\phi}{\partial s}(s,t)-\phi(s,t)\xi(s,t)$ and note that by construction $\beta(s,0)=0$. We will show that $\frac{\partial\beta}{\partial t}(s,t)=\beta(s,t)\eta(s,t)$ so that $\beta$ is identically zero (because of the zero initial condition). We have
\begin{align*}
\frac{\partial\beta}{\partial t}&=\frac{\partial^2\phi}{\partial s\partial t}-\frac{\partial\phi}{\partial t}\xi-\phi\frac{\partial\xi}{\partial t}\\
&=\frac{\partial\phi}{\partial s}\eta+\phi\frac{\partial\eta}{\partial s}-\phi\eta\xi-\phi\frac{\partial\xi}{\partial t}
\end{align*}
where we have used the equation $\partial\phi/\partial t=\phi\eta$ twice. Now we use the identity $\frac{\partial\xi}{\partial t}-\frac{\partial\eta}{\partial s}=[\xi,\eta]$ to get
\begin{align*}\frac{\partial\beta}{\partial t}&=\phi[\eta,\xi]+\frac{\partial\phi}{\partial s}\eta-\frac{\partial\phi}{\partial t}\xi\\
&=\beta\eta\end{align*}
since $[\eta,\xi]=\eta\xi-\xi\eta$.
\end{answer}
\fi


{\bf Paths in $G$ from paths in $\mathfrak{g}$:}

\begin{question}
\begin{enumerate}
\item[(a)]  Let $\gamma\colon[0,1]\to G$ be a path. For each $s\in[0,1]$, by considering the path $\gamma(s)^{-1}\gamma(t)$ show that $A(s)=\gamma^{-1}(s)\dot{\gamma}(s)$ is a tangent vector to $G$ at $1$. That is $A(s)\in\mathfrak{g}$.

\item[(b)] Conversely, let $X\in\mathfrak{g}$ and $g\in G$. Show that $gX$ is a tangent vector to $G$ at $g$. Hence $T_gG=g\mathfrak{g}$.
\end{enumerate}
\end{question}

A consequence of this and the existence/uniqueness of solutions to ODEs is the following lemma which we will use to construct $F$.
\begin{lma}
Let $A\colon[0,1]\to\mathfrak{g}$ be a path in $\mathfrak{g}$. Show that if $\gamma\colon[0,1]\to GL(n,\RR)$ is a path satisfying $\dot{\gamma}(t)=\gamma(t)A(t)$ then $\gamma(t)\in G$ for all $t$.
\end{lma}

\iffalse
\begin{answer}
\begin{enumerate}[(a)]
\item For each $s$, the path $\delta(t)=\gamma(s)^{-1}\gamma(t)$ is a path in $G$ with $\delta(s)=1$. We have $\dot{\delta}(t)=\gamma(s)^{-1}\dot{\gamma}(t)$ so $\dot{\delta}(s)=\gamma(s)^{-1}\dot{\gamma}(s)=A(s)$ is a tangent vector to $G$ at $1$.
\item Conversely if $X\in\mathfrak{g}$ and $g\in G$ then $\gamma(t)=g\exp(tX)\in G$ for all $t$ so $\dot{\gamma}(0)=gX\in T_gG$.
\end{enumerate}

{\small {\bf Out of interest:} To prove the Lemma, we will show that the interval inside $\RR$ on which $\gamma(t)\in G$ is both open and closed, hence it's the whole of $\RR$. Closedness is obvious because $G$ is a closed subset and $\gamma$ is a continuous path. To prove openness suppose that $\gamma(t)\in G$ - we will show that $\gamma(t+\epsilon)\in G$ for $|\epsilon|$ sufficiently small. Note that in local exponential coordinates $\gamma(t)\exp(w_1)\exp(w_2)$ near $\gamma(t)$, (where $w_1\in\mathfrak{g}$ and $w_2\in W_2$ for a complement $W_2$ of $\mathfrak{g}$) a neighbourhood of $\gamma(t)\in G$ is identified with a neighbourhood of $0\in\mathfrak{g}$ and tangent vectors to $G$ are identified with tangent vectors to $\mathfrak{g}$. We define the vector field $B(g,\epsilon)=gA(t+\epsilon)$ depending on $g\in GL(n,\RR)$ and a parameter $\epsilon$. Along $G$ this is tangent to $G$ and hence in the exponential chart it becomes a vector field $\tilde{B}(w_1,w_2,\epsilon)$ which is tangent to $\mathfrak{g}$ along $\mathfrak{g}$ (that is $\tilde{B}(w_1,0,\epsilon)$ has no $W_2$ component). By the existence/uniqueness theorems for ODEs there is a solution to $(\dot{\delta}(\epsilon),0)=\tilde{B}(\delta(\epsilon),0,\epsilon)$ starting at $\delta(0)=0\in\mathfrak{g}$ and the image of this under the exponential chart is a solution $\gamma(t+\epsilon)$ to the original ODE.}
\end{answer}
\fi

\bigskip
\hrule

{\bf The construction of $F(g)$:} 

Let $\gamma\colon[0,1]\to G$ be a path with $\gamma(0)=1$ and $\gamma(1)=g$. Define $A(t)=\gamma(t)^{-1}\dot{\gamma}(t)\in\mathfrak{g}$. Now consider the differential equation
\[\dot{\delta}(t)=\delta(t)f(A(t)),\quad\delta(0)=1\]
for a path $\delta$ in $H$. We call $\delta$ the {\em path associated to $\gamma$}.

\begin{center}\fbox{Define $F(g)=\delta(1)$.}\end{center}

We will show that
\begin{itemize}
\item $\delta(1)$ does not depend on the choice $\gamma$ of path from $1$ to $g$;
\item that $F\colon G\to H$ is a homomorphism;
\item that $F_*=f$.
\end{itemize}
\bigskip
\hrule
\bigskip

{\bf Path-independence of $\delta(1)$:}

Suppose that $\gamma_i$, $i=0,1$ are two choices of path with $\gamma_i(0)=1$ and $\gamma_i(1)=g$ and associated paths $\delta_i$. Since $G$ is simply-connected there is a map $\psi\colon[0,1]\times [0,1]\to G$ such that
\[\psi(0,t)=\gamma_0(t),\quad \psi(1,t)=\gamma_1(t),\quad\psi(s,0)=1,\quad\psi(s,1)=g.\]
\begin{question}
\begin{enumerate}
\item[(a)] Show that $\xi$ and $\eta$ satisfy the Maurer-Cartan equation, where
\[\xi(s,t)=\psi^{-1}\pd{\psi}{s}\in\mathfrak{g},\quad\eta(s,t)=\psi^{-1}\pd{\psi}{t}\in\mathfrak{g}.\]
\item[(b)] If $f\colon\mathfrak{g}\to\mathfrak{h}$ is a Lie algebra homomorphism deduce that $f\xi$ and $f\eta$ also satisfy the Maurer-Cartan equation.
\item[(c)] Deduce that if $f$ is a Lie algebra homomorphism then there exists a map $\phi\colon[0,1]\times[0,1]\to H$ such that $\phi(0,0)=1$ and
\[\pd{\phi}{s}=\phi f(\xi),\quad\pd{\phi}{t}=\phi f(\eta).\]
\item[(d)] Why is $\xi(s,1)=0$? Deduce that $\phi(s,1)$ is independent of $s$ and hence that $\delta_0(1)=\delta_1(1)$.
\end{enumerate}
\end{question}

\iffalse
\begin{answer}
\begin{enumerate}[(a)]
\item
By the Maurer-Cartan theorem, the existence of $\psi$ implies that if
\[\pd{\psi}{s}(s,t)=\psi(s,t)\xi(s,t),\quad\pd{\psi}{t}(s,t)=\psi(s,t)\eta(s,t)\]
then $\xi$ and $\eta$ solve the Maurer-Cartan equation.
\item If $f$ is a Lie algebra homomorphism then
\[\pd{f\xi}{t}-\pd{f\eta}{s}=f\left(\pd{\xi}{t}-\pd{\eta}{s}\right)=f[\xi,\eta]=[f\xi,f\eta].\]
so $f\xi$ and $f\eta$ satisfy the Maurer-Cartan equation.
\item By the Maurer-Cartan theorem if $f\xi$ and $f\eta$ satisfy the Maurer-Cartan equation then the existence of such a $\phi$ follows.
\item Since $\psi(s,1)=g$ we have $\xi(s,1)=\pd{\psi}{s}(s,1)=0$. Hence $f\xi(s,1)=0$ and hence $\pd{\phi}{s}(s,1)=\phi(s,1)f\xi(s,1)=0$ and so $\phi(s,1)$ is independent of $s$. Since $\delta_i$ is the associated path to $\gamma_i$ ($i=0,1$) it is equal to $\psi(i,t)$ and therefore we have proved that $\delta_0(1)=\delta_1(1)$.
\end{enumerate}
\end{answer}
\fi

\newpage

{\bf $F$ is a homomorphism.}

\begin{question}
If $\delta(t)$ is a path in a group then, by differentiating $\delta(t)^{-1}\delta(t)=1$, prove that
\[\dd{(\delta(t)^{-1})}{t}=-\delta(t)^{-1}\dot{\delta}(t)\delta(t)^{-1}.\]
\end{question}

\iffalse
\begin{answer}
Differentiating $\delta(t)^{-1}\delta(t)=1$ with respect to $t$ gives
\[\dd{\delta^{-1}}{t}(t)\delta(t)=-\delta(t)^{-1}\dd{\delta}{t}(t)\]
so
\[\dd{\delta^{-1}}{t}(t)=-\delta(t)^{-1}\dd{\delta}{t}(t)\delta(t)^{-1}.\]
\end{answer}
\fi

Take a path $\gamma$ in $G$ with $\gamma(0)=1$ and $\gamma(1)=g$ and let $\delta$ be the associated path in $H$ (equivalently $\delta^{-1}\dot{\delta}=f(\gamma^{-1}\dot{\gamma})$) and recall that $F(g)=\delta(1)$.

\begin{question}\label{qun:intertwined}
Show that $\delta(t)^{-1}f(X)\delta(t)$ and $f(\gamma(t)^{-1}X\gamma(t))$ are both solutions to the ODE
\[\dot{z}(t)=[z(t),\delta(t)^{-1}\dot{\delta}(t)]\]
with initial condition $z(0)=f(X)$. Deduce that they are equal for all $t$ and hence prove that
\[\delta(1)^{-1}f(X)\delta(1)=f(\gamma^{-1}(1)X\gamma(1)),\]
or, equivalently,
\[F(g)^{-1}f(X)F(g)=f(g^{-1}Xg).\]
\end{question}

\iffalse
\begin{answer}
\begin{align*}
\frac{d}{dt}\left(\delta(t)^{-1}f(X)\delta(t)\right)&=-\delta^{-1}\dot{\delta}\delta^{-1}f(X)\delta+\delta^{-1}f(X)\dot{\delta}\\
&=[\delta^{-1}f(X)\delta,\delta^{-1}\dot{\delta}].
\end{align*}
We also have
\begin{align*}
\frac{d}{dt}\left(f(\gamma^{-1}(t)X\gamma(t)\right)&=-f(\gamma^{-1}\dot{\gamma}\gamma^{-1}X\gamma)+f(\gamma^{-1}X\gamma\gamma^{-1}\dot{\gamma})\\
&=f[\gamma^{-1}X\gamma,\gamma^{-1}\dot{\gamma}]\\
&=[f(\gamma^{-1}X\gamma),f(\gamma^{-1}\dot{\gamma})]\\
&=[f(\gamma^{-1}X\gamma),\delta^{-1}\dot{\delta}]
\end{align*}
Therefore $\delta^{-1}(t)f(X)\delta(t)$ and $f(\gamma(t)^{-1}X\gamma(t))$ are both solutions to the equation
\[\dot{z}=[z,\delta^{-1}\dot{\delta}]\]
with initial condition $\delta^{-1}(0)f(X)\delta(0)=f(\gamma(0)^{-1}X\gamma(0))=f(X)$. By uniqueness we know they are equal for all $t$, in particular for $t=1$, which implies the result.
\end{answer}
\fi

\begin{question}
Suppose that $\alpha,\beta$ are paths in $G$ with $\alpha(0)=\beta(0)=1$ and $\alpha(1)=g$, $\beta(1)=h$. Define the path $\gamma(t)=\alpha(t)\beta(t)$ in $G$ with $\gamma(1)=gh$ and associated paths $u,v,w$ in $H$ satisfying
\begin{align*}
\dot{u}&=uf(\alpha^{-1}\dot{\alpha})\quad\mbox{so }F(g)=u(1)\\
\dot{v}&=vf(\beta^{-1}\dot{\beta})\quad\mbox{so }F(h)=v(1)\\
\dot{w}&=wf(\gamma^{-1}\dot{\gamma})\quad\mbox{so }F(gh)=w(1)
\end{align*}
Prove that $w=uv$ by showing that they both solve the same ODE
\[\dot{z}=zv^{-1}f(\alpha^{-1}\dot{\alpha})v+zf(\beta^{-1}\dot{\beta})\]
with the initial condition $z(0)=1$.
\end{question}

\iffalse
\begin{answer}
We want to show that $w=uv$ and we do this by showing that they both solve the same ordinary differential equation with the same initial conditions. Differentiating:
\begin{align*}
\dot{w}&=wf(\gamma^{-1}\dot{\gamma})\\
&=wf(\beta^{-1}\alpha^{-1}(\dot{\alpha}\beta+\alpha\dot{\beta})\\
&=wf(\beta^{-1}\alpha^{-1}\dot{\alpha}\beta)+wf(\beta^{-1}\dot{\beta})\\
&=wv^{-1}f(\alpha^{-1}\dot{\alpha})v+wf(\beta^{-1}\dot{\beta})
\end{align*}
where we used Question \ref{qun:intertwined} in this last line. We also have
\[\frac{d}{dt}(uv)=\dot{u}v+u\dot{v}=uf(\alpha^{-1}\dot{\alpha}v+uvf(\beta^{-1}\dot{\beta})\]
so we see that both $uv$ and $w$ solve
\[dz/dt=zv^{-1}f(\alpha^{-1}\dot{\alpha})v+zf(\beta^{-1}\dot{\beta})\]
with initial condition $z(0)=1$.
\end{answer}
\fi

{\bf $F_*=f$:}

\begin{question}
Suppose that $\gamma(t)=\exp(tX)$. Prove that the associated path in $H$ is given by $\delta(t)=\exp(tf(X))$. Deduce that $F(\exp X)=\exp(f(X))$ and hence that $f(X)=F_*(X)$.
\end{question}

\iffalse
\begin{answer}
Since $\dot{\gamma}(t)=\gamma(t)X$ we have $A(t)=X$ and hence the associated path is defined by the equation $\dot{\delta}(t)=\delta(t)f(X)$ and $\delta(0)=1$. Thus $\delta(t)=\exp(tf(X))$ by uniqueness of solutions to ODEs. Therefore $F(\exp X)=\delta(1)=\exp(f(X))$ so $F_*=f$.
\end{answer}
\fi

This completes the proof of Lie's theorem. \qedhere

\end{document}