\documentclass[12pt]{article}

\usepackage{parskip,palatino,amsthm,amsmath,amsfonts,amssymb}
\usepackage{multicol}
\usepackage{enumerate}
\usepackage[letterpaper,margin=1in,bottom=0.7in]{geometry}

\newcommand{\dd}[2]{\frac{d #1}{d #2}}
\newcommand{\pd}[2]{\frac{\partial #1}{\partial #2}}
\newcommand{\brf}[2]{\left(\frac{#1}{#2}\right)}
                       % Bracket-frac, e.g. for (n\pi x/L) in Fourier series
\newcommand{\fsin}[1]{\sin\brf{#1 \pi x}{L}}
\newcommand{\fcos}[1]{\cos\brf{#1 \pi x}{L}}
\newcommand{\RR}{\mathbf{R}}
\newcommand{\CC}{\mathbf{C}}
\newcommand{\ZZ}{\mathbf{Z}}
\newcommand{\mk}{\mathfrak}
\newcommand{\mb}{\mathbf}
\newcommand{\OP}{\operatorname}
\newcommand{\MATR}[9]{\left(\begin{array}{cccc}#1 & #2 & \cdots & #3\\ #4 & #5 & \cdots & #6\\ \vdots & \vdots & \ddots & \vdots\\ #7 & #8 & \cdots & #9\end{array}\right)}
\newcommand{\matr}[4]{\left(\begin{array}{cc}#1 & #2\\ #3 & #4\end{array}\right)}
\newcommand{\matt}[9]{\left(\begin{array}{ccc}#1 & #2 & #3\\#4 & #5 & #6\\#7 & #8 & #9\end{array}\right)}
\newcommand{\vect}[3]{\left(\begin{array}{c}#1 \\ #2 \\ #3\end{array}\right)}
\newcommand{\ad}{\OP{ad}}

\newtheorem{thm}{Theorem}
\newtheorem{lma}[thm]{Lemma}

\theoremstyle{definition}
\newtheorem{question}{Question}
\newtheorem{answer}{Answer}

\theoremstyle{remark}
\newtheorem*{rmk}{Remark}

%%%%%%%%%%%%%%%%%% Add extra space before theorems

\begingroup 
\makeatletter 
\@for\theoremstyle:=definition,remark,plain,TheoremNum\do{% 
\expandafter\g@addto@macro\csname th@\theoremstyle\endcsname{% 
\addtolength\thm@preskip\parskip 
}% 
} 
\endgroup 

\include{diagrams}

\title{Sheet 8: (a) The Lie algebra $\mk{g}_2$ and (b) Pentaquarks}
\author{J. Evans}
\date{}

\begin{document}
\maketitle

In the first part of this question sheet we will consider a simple\footnote{...in the technical sense of having no nonzero ideals...} Lie algebra called $\mk{g}_2$. This can be defined explicitly as a Lie algebra of 7-by-7 matrices, but thanks to the general theory we have developed, the only information you will need is its root diagram, which looks like:

\gtwoadj

(The solid line is there to demarcate positive/negative roots). Denote by $\Gamma_{a,b}$ the irreducible representation with highest weight $a\omega_1+b\omega_2$ and by $V_{a,b}$ the underlying vector space of this representation. Note that $\|\omega_1\|^2=1$, $\|\omega_2\|^2=3$ and $\langle\omega_1,\omega_2\rangle=3/2$.

\bigskip

\begin{question}
\begin{enumerate}
\item[(a)] Identify the Weyl group of this Lie algebra.
\item[(b)] Sketch a Weyl chamber on the root diagram.
\item[(c)] Identify the set of positive roots relative to the solid line in the diagram explicitly as linear combinations of $\omega_1$ and $\omega_2$.
\end{enumerate}
\end{question}

\iffalse
\begin{answer}
\begin{enumerate}[(a)]
\item The Weyl group is the dihedral group of order 12 (symmetries of a hexagon) since it is generated by the reflections in root vectors. To see this note that the shorter roots form the vertices of a hexagon. The longer roots bisect the edges and give three of the reflections, the reflections in shorter roots give three more reflections. Those six reflections generate the Weyl group and they are also known to generate the symmetry group of the hexagon.
\item A Weyl chamber is given, for example, by the segment of plane bounded by the rays through $\omega_1$ and $\omega_2$.
\item The positive roots are 

\gtwoadjpos

\end{enumerate}
\end{answer}
\newpage
\fi

\bigskip

\begin{question}
\begin{enumerate}
\item[(a)] Draw the weight diagrams for the irreducible representations
\[\mbox{(i) }\Gamma_{1,0},\qquad\mbox{(ii) }\Gamma_{0,1},\qquad\mbox{(iii) }\Gamma_{2,0},\]
and calculate the multiplicities of the weights that occur using the Freudenthal multiplicity formula or otherwise.
\item[(b)] In each case, what is the dimension of the representation?
\end{enumerate}

{\em Note that since $\mk{g}_2$ is simple, the kernel of any nontrivial representation is zero. In particular, $\mk{g}_2$ injects into $\mk{gl}(V_{1,0})$. This yields the explicit description of $\mk{g}_2$ as a Lie algebra of matrices alluded to at the beginning of the sheet.}
\end{question}

\iffalse
\begin{answer}
The Freudenthal formula says that
\[\left(\|\lambda+\rho\|^2-\|\mu+\rho\|^2\right)n_{\mu}\Gamma_{\lambda}=2\sum_{\alpha\in R^+}\sum_{k\geq 1}\langle\mu+k\alpha,\alpha\rangle n_{\mu+k\alpha}(\Gamma_{\lambda})\]
where $\rho=\frac{1}{2}\sum_{\alpha\in R^+}\alpha$. In the case of $\mk{g}_2$ the positive roots are
\[R^+=\{2\omega_2-3\omega_1,\omega_2-\omega_1,\omega_2,\omega_1,3\omega_1-\omega_2,2\omega_1-\omega_2\}\]
and so $\rho=\omega_1+\omega_2$. We know that the outermost weights in any irreducible representation have multiplicity one.

Note that if we normalise to make $\|\omega_1\|^2=1$ then $\|\omega_2\|=2\cos(\pi/6)=\sqrt{3}$ so $\|\omega_2-\omega_1\|^2=1=\|\omega_2\|^2+\|\omega_1\|^2-2\langle\omega_1,\omega_2\rangle=4-2\langle\omega_1,\omega_2\rangle$ and hence $\langle\omega_1,\omega_2\rangle=3/2$. Note that $\|\rho\|^2=7$.
\begin{enumerate}[(a)]
\item In this case $\lambda=\omega_1$ and the weight diagram looks like this:

\gtwolambdaonezero

and we only need to calculate the weight of $\mu=0$. Freudenthal gives
\[\left(\|\omega_1+\omega_1+\omega_2\|^2-\|\omega_1+\omega_2\|^2\right)n_{\mu}\Gamma_{\lambda}=2\sum_{\alpha\in R^+}\sum_{k\geq 1}\langle\mu+k\alpha,\alpha\rangle n_{\mu+k\alpha}(\Gamma_{\lambda})\]
The coefficient on the LHS is
\[(4+3+6)-7=6.\]
In the sum, the only ways of getting from $\mu=0$ to a weight with nonzero multiplicity along a positive root are along $\alpha\in\{\omega_2-\omega_1,\ \omega_1,\ 2\omega_1-\omega_2\}$ and in each case $k=1$ and $n_{\mu+k\alpha}=1$. We have
\[6n_0(\Gamma_{1,0})=2\left(\langle\omega_2-\omega_1,\omega_2-\omega_1\rangle+\langle\omega_1,\omega_1\rangle+\langle2\omega_1-\omega_2,2\omega_1-\omega_2\rangle\right)=2\times 3\]
so $n_0(\Gamma_{1,0})=1$. Therefore there are seven weight spaces each having dimension one so the representation is seven-dimensional.
\item In this case $\lambda=\omega_2$. Reflecting $\omega_2$ using the Weyl group and taking the convex hull it is clear that the result is the same as taking the convex hull of the adjoint representation. Applying roots it is clear that we will have weights supported in the same places as the adjoint representation, so the weight diagram is the same as the adjoint representation. We still need to check multiplicities but we can take a slight shortcut. Since the adjoint representation has highest weight $\omega_2$ it must contain a copy of this representation. Therefore its multiplicities are upper bounds for the multiplicities in $\Gamma_{0,1}$. By considering the $\mk{sl}(2,\CC)$ subalgebras for the short roots it is clear that the weights with length one (i.e. $\omega_1$ and its Weyl reflections) must have nonzero multiplicity and hence the multiplicity is precisely one. The only question is over the central weight. The coefficient on the LHS of Freudenthal is
\[\|2\omega_2+\omega_1\|^2-\|\omega_1+\omega_2\|^2=19-7=12.\]
Since all other multiplicities are one, $k=1$ and $\mu=0$, the sum on the RHS is the sum of squared lengths of positive roots, which is $3+9=12$. Therefore
\[12n_0(\Gamma_{0,1})=2\times 12\]
and $n_0=2$. So this is the adjoint representation!
\item In this case the weight diagram (obtained by reflecting $2\omega_2$ under the Weyl group and applying all possible roots) is

\gtwolambdatwozero

The outermost weights are all one. We calculate the weights at $\omega_1$ and at $0$ (all others are determined by Weyl symmetry). For $\omega_1$ we have
\[\|2\omega_1+\rho\|^2-\|\omega_1+\rho\|^2=8\]
and in the sum we have $k=1$ and $\alpha$ runs over the short roots and over $2\omega_2-3\omega_1$ so we get
\[2\left(\langle\omega_2,\omega_2-\omega_1\rangle+\langle 2\omega_1,\omega_1\rangle+\langle 3\omega_1-\omega_2,2\omega_1-\omega_2)+\langle 2\omega_2-2\omega_1,2\omega_2-3\omega_1\rangle\right)\]
or
\[2(3/2+2+3/2+3)=16\]
Therefore
\[8n_{\omega_1}=16\ \Rightarrow n_{\omega_1}=2.\]
For $\mu=0$ we get
\[\|2\omega_1+\rho\|^2-\|\rho\|^2=14\]
and in the sum we now have:
\begin{itemize}
\item a $k=1$, $n_{\mu+\alpha}=2$ term for each short positive root $\alpha$, giving a total contribution of $2\times 2\times(1+1+1)=12$ to the sum,
\item a $k=1$, $n_{\mu+\alpha}=1$ term for each long positive root $\alpha$, giving a total contribution of $2\times(3+3+3)=18$ to the sum,
\item a $k=2$, $n_{\mu+2\alpha}=1$ term for each short positive root $\alpha$, giving a total contribution of $2\times(2+2+2)=12$ to the sum,
\end{itemize}
so Freudenthal implies
\[14n_0=12+18+12=42\ \Rightarrow\ n_0=3.\]
The total dimension of the representation is $3+2\times 6 +1\times 12=27$.
\end{enumerate}
\end{answer}
\newpage
\fi

\bigskip

\begin{question}\label{qun:g2decom}
Decompose the following representations into their irreducible parts:
\[\mbox{(a) }\OP{Sym}^2(\Gamma_{1,0}),\qquad\mbox{(b) }\Lambda^3\Gamma_{1,0}.\]
Deduce that there are: (a) a unique (up to rescaling) quadratic form and (b) a unique (up to rescaling) skew-symmetric trilinear form on the vector space $V_{1,0}$ which are invariant under the action of $\mk{g}_2$.

{\em In fact: (a) This quadratic form turns out to be nondegenerate so this tells us that the image of this 7-dimensional representation lands in $SO(7,\CC)\subset GL(7,\CC)$. (b) This trilinear form is related to the algebra of octonions; one way to define $\mk{g}_2$ is as the Lie algebra of derivations of the octonion algebra $\mb{O}$. In this sheet we have effectively proved that $\mk{g}_2\subset\mk{der}(\mb{O})$.}
\end{question}

\iffalse
\begin{answer}
\begin{enumerate}[(a)]
\item If we label the vertices of $\Gamma_{1,0}$ thus

\gtwolabel

then the weight diagram for $\OP{Sym}^2(\Gamma_{1,0})$ is

\gtwolabeltwo

where we have written on the weight vectors for the 1-dimensional weight spaces and written on the multiplicity otherwise. For instance at $\omega_1$ the weight space is spanned by $yp$ and $xz$ and at the origin the weight space is spanned by $xa$, $yb$, $zc$ and $p^2$.

Stripping off a copy of $\Gamma_{2,0}$ we are left with a copy of the trivial 1-dimensional representation so $\OP{Sym}^2(\Gamma_{1,0})=\Gamma_{2,0}\oplus\CC$. In particular (since $\Gamma_{1,0}\cong\Gamma_{1,0}^*$ by symmetry of the weight diagram) this tells us that there is an invariant quadratic polynomial on $V_{1,0}$.
\item The weight diagram of $\Lambda^3\Gamma_{1,0}$ is

\gtwolambdathree

For example at $2\omega_1$ the weight vector is $x\wedge y\wedge z$, at $\omega_2$ the weight vector is $p\wedge x\wedge y$, at $\omega_1$ the weight vectors are $x\wedge z \wedge p$, $z\wedge c\wedge y$ and $x\wedge a\wedge y$ and the weight space at the origin is spanned by
\[p\wedge x\wedge a,\ p\wedge y\wedge b,\ p\wedge z\wedge c,\ x\wedge z\wedge b,\ y\wedge a\wedge c.\]

Stripping off a copy of $\Gamma_{2,0}$ we are left with $\Gamma_{1,0}\oplus\CC$. The trivial 1-dimensional subrepresentation is spanned by an invariant 3-form as required.
\end{enumerate}
\end{answer}
\newpage
\fi

\bigskip

\begin{question}[Pentaquarks]
A pentaquark is a hypothetical particle which is a bound state of four quarks and an antiquark. These are believed to be so unstable that, even if it were possible to produce them, they decay too quickly to observe (an all-too-convenient explanation for why we haven't seen them yet...). Classify pentaquarks made from $u,d,s$ quarks/antiquarks into multiplets according to their transformation properties under the group $SU(3)$.
\end{question}

\iffalse
\begin{answer}
There are two simple roots: $2\omega_2-3\omega_1$ (length $\sqrt{3}$) and $2\omega_1-\omega_2$ (length $1$). The angle between them is $5\pi/6$ so the Dynkin diagram is

\gtwodynkin

\end{answer}
\fi

\end{document}